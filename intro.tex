\section{Introduction}
\label{sec:intro}

Let's cite a paper~\cite{DBLP:conf/aaai/PangLGXSC19}.

In summary, the main contributions in this paper are as follows:
\begin{itemize}
	\item We employs a automatic data augmentation framework using Large Language Model \((LLM)\) as a knowledge source and a extra content supplement to linearize relevant information and possible continuation from LLM as texts, then inject them into original contexts. 
	\item We develop a series of prompt templates designed for interacting with ChatGPT to acquire comprehensive explanations of numerous entities. These templates ensure that the formats of the responses provided by ChatGPT are highly parseable and well-structured.
	
\end{itemize}

%The introduction should briefly place the study in a broad context and highlight why it is important. It should define the purpose of the work and its significance. The current state of the research field should be reviewed carefully and key publications cited. Please highlight controversial and diverging hypotheses when necessary. Finally, briefly mention the main aim of the work and highlight the principal conclusions. As far as possible, please keep the introduction comprehensible to scientists outside your particular field of research. Citing a journal paper \cite{ref-journal}. Now citing a book reference \cite{ref-book1,ref-book2} or other reference types \cite{ref-unpublish,ref-communication,ref-proceeding}. Please use the command \citep{ref-thesis,ref-url} for the following MDPI journals, which use author--date citation: Administrative Sciences, Arts, Econometrics, Economies, Genealogy, Humanities, IJFS, Journal of Intelligence, Journalism and Media, JRFM, Languages, Laws, Religions, Risks, Social Sciences, Literature.
%%%%%%%%%%%%%%%%%%%%%%%%%%%%%%%%%%%%%%%%%%
